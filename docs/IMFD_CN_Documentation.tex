\documentclass[11pt]{article}
\usepackage{amsmath,amsfonts,amssymb}
\usepackage{tikz}
\usepackage{graphicx}
\usepackage[utf8]{inputenc}
\usepackage[french]{babel}
\usepackage{geometry}
\geometry{margin=2.5cm}

\title{Méthodes Numériques pour l'Équation de Black-Scholes}
\author{Projet PAP}
\date{\today}

\begin{document}

\maketitle

\section*{Introduction}
Ce document présente une explication détaillée des méthodes numériques utilisées dans la résolution de l'équation de Black-Scholes pour options européennes :

\begin{itemize}
    \item \textbf{Méthode implicite (IMFD)} sur l'EDP réduite
    \item \textbf{Méthode de Crank-Nicholson (CN)} sur l'EDP complète
\end{itemize}

Chaque schéma est accompagné d'une description mathématique, d'un algorithme, et d'une illustration graphique.

\section{EDP de Black-Scholes}
L'équation de Black-Scholes complète s'écrit :
\begin{equation}
    \frac{\partial u}{\partial t} + \frac{1}{2}\sigma^2 s^2 \frac{\partial^2 u}{\partial s^2} + rs \frac{\partial u}{\partial s} - ru = 0
\end{equation}
avec $u(T, s) = \max(s-K, 0)$ pour une option call.

\section{Méthode IMFD (EDP réduite)}
\subsection*{Forme réduite}
On effectue un changement de variable (log, temps inversé, etc.) pour obtenir une EDP de la forme :
\begin{equation}
    \frac{\partial u}{\partial t} = a \frac{\partial^2 u}{\partial s^2}
\end{equation}
avec $a = -\frac{1}{2}\sigma^2$.

\subsection*{Discrétisation implicite}
Pour chaque point $u_i^n$ de la grille :
\begin{equation}
    -\alpha u_{i-1}^{n+1} + (1 + 2\alpha) u_i^{n+1} - \alpha u_{i+1}^{n+1} = u_i^n
\end{equation}
avec $\alpha = \frac{a \Delta t}{(\Delta s)^2}$.

\begin{tikzpicture}[scale=1.2]
    \draw[step=1cm,gray,very thin] (0,0) grid (5,4);
    \foreach \x in {1,...,4} \draw (\x,0) node[below]{$i=\x$};
    \foreach \y in {1,...,3} \draw (0,\y) node[left]{$n=\y$};
    \draw[->, thick, blue] (1,1) -- (1,2);
    \node at (1,2.2) {\small$u_i^{n+1}$};
    \node[draw=blue, circle, minimum size=0.4cm, inner sep=0pt] at (1,2) {};
\end{tikzpicture}

\subsection*{Résolution}
- Résolution d'un système tridiagonal à chaque pas de temps \rightarrow algorithme de Thomas

\section{Méthode Crank-Nicholson (EDP complète)}
\subsection*{Schéma semi-implicite}

\begin{equation}
    \frac{u_i^{n+1} - u_i^n}{\Delta t} = \frac{1}{2} \left[ A u^{n+1} + A u^n \right]
\end{equation}

Avec $A$ la matrice tridiagonale dérivée de l'EDP.

\subsection*{Matrice tridiagonale}
\begin{equation*}
A = \begin{bmatrix}
    b_1 & c_1 &        &        &        \\
    a_2 & b_2 & c_2    &        &        \\
        & \ddots & \ddots & \ddots &    \\
        &        & a_{n-1} & b_{n-1} & c_{n-1} \\
        &        &        & a_n    & b_n
\end{bmatrix}
\end{equation*}

\section{Conditions aux bords}
\begin{itemize}
    \item $u(t, 0) = 0$ pour un call
    \item $u(t, L) = L - K e^{-r(T - t)}$
\end{itemize}

\section{Conclusion}
La comparaison graphique des deux méthodes permet d'évaluer :
\begin{itemize}
    \item la stabilité des schémas
    \item la convergence numérique
    \item l'erreur absolue entre deux formulations
\end{itemize}

\end{document}